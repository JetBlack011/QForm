\section{Computing Intersection Forms}

The intersection form of a given multisection diagram can be computed and expressed entirely in terms of an algebraic multisection and the intersection pairing on $H_1(\Sigma_g)$. That is to say, when computing intersection forms, we don't need to think of curves on a genus $g$ surface, but rather of algebraic data on $H_1(\Sigma_g)$. The reason for this comes from the nature of algebraic multisections; since $L_i + L_{i + 1} = H_1(\Sigma_g) = \Z^{2g}$ (analogous to the condition that the corresponding cut systems form a Heegaard diagram for a $\ZHS^3$) and each $L_i$ has dimension $g$, it must be that the elements of the cut systems $\overline{\alpha}_i, \overline{\alpha}_{i + 1}$ giving rise to $L_i,L_{i + 1}$ form a basis of $H_1(\Sigma_g)$. In particular, we can write the remaining homology classes as a $\Z$-linear combination of elements of $\overline{\alpha}_i, \overline{\alpha}_{i + 1}$. For our purposes, this means we can fix $L_1,L_2$ so that $L_1 \oplus L_2 = H_1(\Sigma_g)$.

Recall, for an algebraic multisection $\{L_i\}_{i = 1}^n$ yielding the $4$-manifold $X$, we have that $H_2(X)$ is given by the kernel of the map 
\begin{align*}
	\rho: L_1 \oplus L_2 \oplus \dots \oplus L_n &\to H_1(\Sigma_g)\\
	(\overline{\alpha}_1, \overline{\alpha}_2, \dots, \overline{\alpha}_n) &\mapsto \overline{\alpha_1} + \dots + \overline{\alpha_n}
\end{align*} 
where the cut system $\overline{\alpha}_i$ corresponds to the Lagrangian $L_i$. Additionally, the intersection form \[Q([\underline{x}],[\underline{y}]) = \sum_{1 \leq i < j \leq n} \left< \overline{x}_i, \overline{y}_i \right>\] is given entirely in terms of the basis of $H_2(X) = \ker\rho$ and the intersection pairing on $H_1(\Sigma_g)$. Thus, the intersection pairing gives us all of the information we need to determine the remaining classes in terms of this basis.

% \subsection{An Example}
% Here, we will go through a non-trivial example of computing the intersection form given the intersection data of some genus $2$ trisection. In particular, suppose we have the algebraic trisection $\{L_i\}_{i = 1}^3$ given by the trisection diagram $(\Sigma_2, \overline{\alpha}, \overline{\beta}, \overline{\gamma})$. Suppose we also have the intersection pairing $\left< , \right>$ on $H_1(\Sigma_2)$ given by this diagram. 

\subsection{Algorithm for Arbitrary Intersection Forms}

Here, we describe the algorithm used to compute the matrix for an intersection form given a genus $g$ algebraic multisection $\{L_i\}_{i = 1}^n$ and the intersection pairing on $H_1(\Sigma_g)$. Suppose that $\overline{\alpha}_i$ generates $L_i$ for all $i$. 
\begin{enumerate}
	\item First, we must find a basis for $\ker\rho$ where $\rho$ is the same map as above. This is as easy as finding an expression for $\overline{\alpha}_i \subseteq H_1(\Sigma_g)$ for $i > 2$ in terms of a $\Z$-linear combination of \[\overline{\alpha} = \overline{\alpha}_1 = \{\alpha_1,\dots,\alpha_g\}, \quad \overline{\beta} = \overline{\alpha}_2 = \{\beta_1,\dots,\beta_g\}.\] Fix some $\gamma \in \overline{\alpha}_i$ for some $i > 2$. We can write \[\gamma = a_1\alpha_1 + \dots + a_g\alpha_g + b_1\beta_1 + \dots + b_g\beta_g\] where $a_1,\dots,a_g,b_1,\dots,b_g \in \Z$. Then for some $\alpha \in \overline{\alpha}$, we have that since $L_1$ is a Lagrangian subspace of $H_1(\Sigma_g)$,
	\begin{align*}
		\left< \gamma, \alpha \right> &= \left< a_1\alpha_1 + \dots + a_g\alpha_g + b_1\beta_1 + \dots + b_g\beta_g, \alpha_1 \right>\\
									  &= \left< b_1\beta_1 + \dots + b_g\beta_g, \alpha \right>\\
									  &= \left< \beta_1, \alpha \right>b_1 + \dots + \left< \beta_g, \alpha \right>b_g.
	\end{align*}
	Since each intersection pairing is an integer we know, we have an equation for the indeterminates $b_1,\dots,b_g$. Moreover, since we can repeat this for all $\alpha \in \overline{\alpha}$, we have $g$ such equations. Similarly, we can repeat the process using each $\beta \in \overline{\beta}$, so that we have $g$ equations for $a_1,\dots,a_g$. We can thus represent these linear system of equations as follows:
	\begin{align*}
		\begin{bmatrix}
			0 & \dots & 0 & \<\alpha_1, \beta_1\> & \dots & \<\alpha_1, \beta_g\>\\
			\vdots & \ddots & \vdots & \vdots & \ddots & \vdots\\
			0 & \dots & 0 & \left< \alpha_g, \beta_1 \right> & \dots & \left< \alpha_g, \beta_g \right>\\
			\left< \beta_1, \alpha_1 \right> & \dots & \left< \beta_1, \alpha_g \right> & 0 & \dots & 0\\
			\vdots & \ddots & \vdots & \vdots & \ddots & \vdots\\
			\left< \beta_g, \alpha_1 \right> & \dots & \left< \beta_g, \alpha_g \right> & 0 & \dots & 0
			\end{bmatrix} \begin{pmatrix} b_1\\ \vdots \\ b_g \\ a_1 \\ \vdots \\ a_g \end{pmatrix} = \begin{pmatrix} \left< \gamma, \beta_1 \right> \\ \vdots \\ \left< \gamma, \beta_g \right> \\ \left< \gamma, \alpha_1 \right> \\ \vdots \\ \left< \gamma, \alpha_g \right> \end{pmatrix} 
	\end{align*}
	Since $(\Sigma_g, \overline{\alpha}, \overline{\beta})$ forms a Heegaard diagram for an $\ZHS^3$, the two nonzero blocks in the coefficient matrix are of full rank (note that if we denote the upper right block and lower left block as $\left< \overline{\alpha}, \overline{\beta} \right>$ and $\left< \overline{\beta}, \overline{\alpha} \right>$, respectively, then $\left< \overline{\beta}, \overline{\alpha} \right> = -\left< \overline{\alpha}, \overline{\beta} \right>^T$). It follows that the whole matrix is invertible, and so we can solve for $a_1,\dots,a_g,b_1,\dots,b_g$ in a succinct, systematic way. Repeating this for each $\gamma \in \overline{\alpha_i}$ for all $i > 2$ gives each homology class in terms of our chosen basis of $\overline{\alpha}, \overline{\beta}$. However, there is, of course, the chance that these coefficients will not be strictly integers; some may be rational. In this case, it is sufficient for our purposes to take the least common multiple $\ell$ of the denominators of the coefficients, and consider \[\ell \gamma = \ell(a_1\alpha_1 + \dots + a_g\alpha_g + b_1\beta_1 + \dots + b_g\beta_g).\] 

	It is now trivial to find the kernel of our map $\rho$. In particular, if we denote $\{a_k^{i,j}\}_{k = 1}^{g},\{b_k^{i,j}\}_{k = 1}^g$ as the sets of coefficients (adjusted by some $\ell^{i,j}$) for $\gamma_{i,j} \in \overline{\alpha}_i$ for $2 < i \leq n$ and $1 \leq j \leq g$, and if we denote \[e_{i,j} = \left( \sum_{k = 1}^{g} a^{i,j}_k\alpha_k \right) + \left( \sum_{k = 1}^g b^{i,j}_{k}\beta_k \right) - \ell^{i,j}\gamma_{i,j},\] then we have that 
	 \[\sbox0{$
	 	\begin{array}{ccc}
			e_{3,0}, & \dots, & e_{3,g},\\
			\vdots & \ddots & \vdots\\
			e_{n,0}, & \dots, & e_{n,g}
	 	\end{array}
	 $}
	 \ker\rho = \Z\mathopen{\resizebox{1.2\width}{\ht0}{$\Bigg\<$}}
		\usebox{0}
	\mathclose{\resizebox{1.2\width}{\ht0}{$\Bigg\>$}}\]
	 % \[\sbox0{$
	 % 	\begin{array}{ccc}
	 % 		\left(\sum\limits_{k = 1}^g a^{3,0}_k\alpha_k\right) + \left(\sum\limits_{k = 1}^g b^{3,0}\beta^{3,0}_k\right) - \ell^{3,0}\gamma_{3,0}, & \dots, & \left( \sum\limits_{k = 1}^g a_k^{g,0}\alpha_k \right) + \left( \sum\limits_{k = 1}^{g} b_k^{g,0}\beta_k \right) - \ell^{g,0}\gamma_{g,0},\\
	%%  		\vdots & \ddots & \vdots\\
	% 		\left( \sum\limits_{k = 1}^g a^{3,g}_k\alpha_k \right) + \left( \sum\limits_{k = 1}^g b^{3,g}_k\beta_k \right) - \ell^{3,g}\gamma_{3,g}, & \dots, & \left( \sum\limits_{k = 1}^g a^{g,g}_k\alpha_k \right) + \left( \sum\limits_{k = 1}^g b^{g,g}_k\beta_k \right) - \ell^{g,g}\gamma_{g,g}
	% 	\end{array}
	% $}
	% \Z\mathopen{\resizebox{1.2\width}{\ht0}{$\Bigg\<$}}
	% 	\usebox{0}
	% \mathclose{\resizebox{1.2\width}{\ht0}{$\Bigg\>$}}\]$
	\item Now that we have a basis for our kernel in terms of $\overline{\alpha},\overline{\beta}$, it is straight forward to express the intersection form $Q$ as a matrix in terms of these basis elements. This is because the sums for $Q$ on each $e_{i,j}$ is taken over $\Z$-linear combinations of intersection pairings of our basis elements, hence is something we can compute. So, if we reindex this basis by row so that $\{e_i\}_{i=1}^{g(n - 2)}$ generates $\ker\rho$, then the matrix representation of $Q$ is $\begin{pmatrix} Q(e_i,e_j) \end{pmatrix}_{i,j}$.
\end{enumerate}
