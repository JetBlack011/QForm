\subsection{Invariants}

\begin{definition}[Intersection Form]
	For some closed oriented topological $4$-manifold $M$, the intersection form $Q_M: H^2(M) \times H^2(M) \to \Z$ is a symmetric unimodular bilinear form given by the cup product: \[Q_M(a,b) = \left< a \smile b, [M] \right>.\] If $M$ is smooth, then $Q_M(a,b) = \int_M \alpha \wedge \beta$ where $a$ and $b$ are represented by $2$-forms $\alpha$ and $\beta$ in $H^2_{dR}(M;\R)$. 
\end{definition}

Dually, this is equivalent to taking the sums of oriented intersection numbers of $2$-cycles, thus we also have such a bilinear pairing on $H_2(M)$. Once we understand how to compute the homology of $M$ given a $(g,k)$-multisection, it is possible to give $Q_M$ in terms of the intersection pairing $\left< \cdot,\cdot \right>_{\Sigma}$ on $H_1(\Sigma)$.

Let $(\Sigma,X_1,\dots,X_n)$ denote a $(g,k)$-multisection of a closed oriented $4$-manifold $X$, and consider the handlebodies $H_{i,j} = X_i \cap X_j$. Let $\iota_i: \Sigma \to H_{i,i+1\Mod n}$ be the inclusion map, so that we have the induced inclusions $\iota_{i*}: H_1(\Sigma) \to H_1(H_{i,i+1 \Mod n})$ for all $1 \leq i \leq n$. Then suppose $L_i = \ker(\iota_{i*})$, so that $L_i$ is a Lagrangian subspace of $H_1(\Sigma)$ generated by any choice of oriented defining curves for $H_{i,i+1 \Mod n}$. 

\begin{theorem}
	The homology of $X$ with coefficients in $\Z$ canonically identifies with the homology of the following complex \cite{Feller}:
	\[0 \to \bigoplus_{i = 1}^n (L_{i - 1} \cap L_i) \xrightarrow{\partial_2} \bigoplus_{i = 1}^n L_i \xrightarrow{\partial_1} H_1(\Sigma) \xrightarrow{\partial_0} \Z \to 0\]
	where \[\partial_2\left((x_i)_{1 \leq i \leq n}\right) = \left( (x_i - x_{i + 1})_{1 \leq i \leq n} \right) \quad \text{and} \quad \partial_1\left( (x_i)_{1 \leq i \leq n} \right) = \left(\bigoplus_{i = 1}^n \iota_{i*}\right)\left( (x_i)_{1 \leq i \leq n} \right) = \sum_{i = 1}^n x_i.\] Denoting $\partial_1$ by $\iota$, we have that $H_2(X) \iso \ker(\iota)$\footnote{todo: Is this only true in the $(g,0)$ case?}.
\end{theorem}

\begin{theorem}
	Suppose $c_1 = \left( (x_i)_{1 \leq i \leq n} \right)$ and $c_2 = \left( (y_i)_{1 \leq i \leq n} \right)$ are elements of $H_2(X)$ with $(x_i)_{1 \leq i \leq n},(y_i)_{1 \leq i \leq n} \in \bigoplus_{i} L_i$. Then the intersection form \[Q_X(c_1,c_2) = \sum_{1 \leq i < j \leq n} \left< x_i,y_j \right>_\Sigma\] It follows that $Q_X$ is represented as a bilinear form by the symmetric unimodular matrix $Q = \left(Q_X(e_i,e_j)\right)_{ij}$ where $(e_i)_{1 \leq i \leq g(n - 2)}$ generate $H_2(X)$. That is, $Q_X(c_1,c_2) = c_1^TQc_2$ for all $c_1,c_2 \in H_2(X)$.\footnote{todo: Prove that $Q$ is symmetric?}
\end{theorem}

\subsection{Motivation and Background}

In this paper, we are concerned with classifying the intersection forms arising from $(2,0)$-multisections. \footnote{todo: State big theorems and relevance}

\section{Intersection Forms Arising from $(2,0)$-Multisections}
\subsection{Standard Examples}

\subsection{Computing Intersection Forms}

\begin{theorem}
	For cut systems $\overline{\alpha}_1, \dots, \overline{\alpha}_n$ defining a $(g,0)$-multisection of a manifold $X$, the intersection form $Q_X$ of $X$ can be computed by the following algorithm:
	\begin{enumerate}
		\item We can write \[\gamma_{i,j} = \left( \sum_{k = 1}^g a_k^{i,j} \alpha_k \right) + \left( \sum_{k = 1}^g b_k^{i,j}\beta_k \right)\] for $\gamma_{i,j} \in \overline{\alpha_i}$ and for $a^{i,j}_k,b^{i,j}_k \in \Z$. Then set 
	\end{enumerate}
\end{theorem}


Generating examples using symplectic structure and generators of symplectic group
geometric realization

\subsubsection*{Requisite Results}
We first begin with some basic results from linear symplectic geometry over (free) $\Z$-modules. Consider the symplectic module $(V = \Z^{2g}, \omega)$\footnote{todo: Background? How much of this has been covered already?} .
\begin{theorem}
	For any Lagrangian subspace $L_1$ of $V$, there exists a \textit{complementary} or \textit{dual} Lagrangian subspace $L_2$ such that $L_1 \oplus L_2 = V$. Moreover, a choice of basis $x_1,\dots,x_g$ of $L_1$ determines a (non-canonical) dual basis $y_1,\dots,y_g$ for a complement $L_2$ by $\omega(x_i, y_j) = \delta_{ij}$. Taken together, these bases form a \textit{symplectic basis} for $V$. 
\end{theorem}

\begin{proposition}
	\label{prp:lagrangian_unimodular}
	Observe, any Lagrangian subspace $L$ of $V$ spans a sublattice of a unimodular $g$-dimensional lattice $\Lambda$ (since the dimension of any Lagrangian subspace is $g/2 = g$). If $L$ is Lagrangian, then $L = \Lambda$; that is, $L$ is full\footnote{Right word?}.
\end{proposition}
\begin{proof}
	Suppose for contradiction that $L \subsetneq \Lambda$ and that $L$ is Lagrangian. Since $\Lambda$ is unimodular, there exists a basis $a_1,\dots,a_g$ of $\Lambda$ such that the Gram matrix of $\Lambda$ in terms of this basis has determinant $\pm 1$. Of course, there must exist some $a_i \notin L$, as otherwise $L = \Lambda$. Moreover, there exists some $\lambda \in \Z$ such that $\lambda a_i \in L$. Thus, since $\omega(x, \lambda a_i) = 0$ for all $x \in L$, we can conclude that $\omega(x, a_i) = 0$, and thus $a_i \in L^\omega$. But this is a contradiction, as since $a_i \notin L$ but $a_i \in L^\omega$, we have that $L \neq L^\omega$, and $L$ is not Lagrangian.
\end{proof}

\begin{corollary}
	For any two Lagrangian subspaces $L,L'$ of $V$, there exists some $T \in \GL_{2g}(\Z)$ such that $T(L) = L'$.
\end{corollary}

\begin{proposition}
	If $L_1,L_2$ and $L_1',L_2'$ are pairs of complementary Lagrangian subspaces, then there exists some $T \in \Sp_{2g}(\Z)$ such that $T(L_1) = L_1'$ and $T(L_2) = L_2'$.
\end{proposition}
\begin{proof}
	Fix a symplectic basis $x_1,\dots,x_g,y_1,\dots,y_g$ and $x_1',\dots,x_g',y_1',\dots,y_g'$ for $L_1 \oplus L_2$ and $L_1' \oplus L_2'$, respectively. Then since each of $L_1,L_2,L_1'$, and $L_2'$ are unimodular, the map $T$ with $x_i \mapsto x_i'$ and $y_i \mapsto y_i'$ is clearly symplectic, as $T$ is a linear isomorphism and $T^*\omega = \omega$.
\end{proof}

\begin{corollary}
	\label{cor:symplectic_existence}
	Between any two Lagrangian subspaces $L,L'$ of $V$, there exists a symplectic map $T: L \to L'$. Conversely, for any symplectic transformation $T \in \Sp_{2g}(\Z)$, the image of some Lagrangian subspace $L$ under $T$ is again Lagrangian.
\end{corollary}

\begin{definition}[Geometric Realization]
	A \textit{geometric realization} of a symplectic basis $x_1,\dots,x_g,y_1,\dots,y_g$ is an ordered sequence $(\alpha_1,\dots,\alpha_g,\beta_1,\dots,\beta_g)$ of oriented simple closed curves such that
	\begin{enumerate}
		\item $[\alpha_i] = x_1$ and $[\beta_i] = y_i$ for all $1 \leq i \leq g$.
		\item $\alpha_i \cdot \alpha_j = \beta_i \cdot \beta_j = 0$ and $\alpha_i \cdot \beta_j = \delta_{i,j}$ for all $1 \leq i,j \leq g$.
	\end{enumerate}
	Informally, this means that the curves on $\Sigma_g$ and their algebraic intersection numbers agrees with their homology classes and intersection pairing.
\end{definition}

\begin{definition}[Primitive Element]
	An element $x \in H_1(\Sigma_g)$ is \textit{primitive} if it cannot be written as $x = \lambda x'$ for some $x' \in H_1(\Sigma)$ and $n \geq 2$. 
\end{definition}

\begin{lemma}
	\label{lem:primitive_element}
	A nonzero $x \in H_1(\Sigma_g)$ can be written as $x = [\gamma]$ for some simple closed curve $\gamma$ if and only if $x$ is primitive.
\end{lemma}
\begin{proof}
	\footnote{Cite Putman}
\end{proof}

\begin{theorem}
	Every symplectic basis for $H_1(\Sigma_g)$ has a geometric realization.
\end{theorem}
\begin{proof}
	This is a consequence of \autoref{lem:primitive_element} and \autoref{prp:lagrangian_unimodular} --- A basis of a unimodular lattice necessarily consists of primitive elements, hence these basis elements have a geometric realization. An alternative, full proof can be seen in \footnote{Cite Putman}.
\end{proof}
