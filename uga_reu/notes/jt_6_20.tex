\section{Week of 6/20: Mainly Lots of Examples}

\subsection*{Goals and Overview}
Two of our main goals this week were to
\begin{enumerate}
	\item Put together the Heegaard diagrams for integer homology spheres we had found to form new multisection diagrams for which we can compute invariants (e.g. intersection form).
	\item Find methods for generating many such examples quickly, either in the form of Heegaard diagrams or multisections.
\end{enumerate}

These goals led to fruitful examples and inquiries into the structures we're dealing with; in particular, we now better understand the construction of intersection forms for multisections and the linear symplectic geometry underlying our ideas.

\subsection{Investigating a Nontrivial $4$-section}

Until now, we have found a number of families of Heegaard diagrams representing integer homology spheres. This week, we managed to string them together in such a way that they formed an interesting $4$-section diagram.
\begin{figure}[H]
    \centering
    \incfig[.8]{4-section}
    \caption{A nontrivial $4$-section}
    \label{fig:4-section}
\end{figure}
If we denote $M_{\alpha,\beta} = \left(\left< \alpha_i, \beta_i \right>\right)_{i,j}$ (and similarly for the other cut systems), we have the following intersection data:
\begin{align*}
	M_{\alpha,\beta} &= \begin{bmatrix} 1 & 0\\0 & 1 \end{bmatrix}, & M_{\alpha,\gamma} &= \begin{bmatrix} 1 & 0\\0 & 1 \end{bmatrix} & M_{\alpha,\delta} &= \begin{bmatrix} -1 & -1\\0 & -1 \end{bmatrix} \\
	&& M_{\beta,\gamma} &= \begin{bmatrix} -1 & 0\\0 & -1 \end{bmatrix} & M_{\beta,\delta} &= \begin{bmatrix} 0 & -1\\1 & 0 \end{bmatrix} \\
	&&&& M_{\gamma,\delta} &= \begin{bmatrix} -1 & 0\\1 & 1 \end{bmatrix} 
\end{align*}
One can verify that these matrices are unimodular, hence each ``adjacent'' pair of cut systems is a Heegaard diagram for an integer homology sphere. In this case, each of these diagrams represent $S^3$. Using the algorithm described before, we compute the intersection form to be \footnote{todo: Something is wrong here, this matrix is not symmetric; check intersection numbers and program for bugs} \[Q = \begin{bmatrix} 1 & 0 & -1 & 1\\0 & 1 & 0 & 1\\1 & 0 & 0 & 1\\-1 & -1 & 1 & -1 \end{bmatrix} \] This form is indefinite odd with signature $-2$.

\subsection{Brieskorn Spheres and The Poincar\'e Homology Sphere}

Having explored a few nontrivial multisections, we were retrospectively curious about nontrivial Heegaard diagrams. Multisections are defined so as to allow us to ``fill in'' the holes left after gluing with fake $4$-balls, namely those whose boundary is an integer homology sphere. All of our examples so far used only copies of $S^3$ as these boundaries, that is, our adjacent cut systems were Heegaard diagrams for $S^3$. What examples are there where this is not the case?

This question led us to investigate the homology spheres that are already well known, as well as their Heegaard splittings. In particular, we focused on a family of homology spheres known as \textit{Brieskorn spheres}.

\begin{definition}[Brieskorn Manifold]
	Let $\Sigma(p,q,r)$ be the smooth, compact $3$-manifold obtained by intersecting the complex hypersurface \[x_1^p + x_2^1 + x_3^r = 0\] with the unit sphere $|x_1|^2 + |x_2|^2 + |x_3|^2 = 1$ with $p,q,r$ integers $\geq 2$. For some fixed $p,q,$ and $r$, we call $\Sigma(p,q,r)$ a Brieskorn manifold.\footnote{todo: Cite Milnor}
\end{definition}

\begin{theorem}
	If $\Sigma = \Sigma(p,q,r)$ where $p,q,r$ are pairwise coprime, then $H_1(\Sigma)$ is trivial. That is, $\Sigma$ is an $\ZHS^3$. We call these \textit{Brieskorn spheres}. If one of $p,q,r$ is $1$, then $\Sigma$ is homeomorphic to $S^3$.
\end{theorem}

\begin{example}
	One of the most important examples of a Brieskorn sphere is $\Sigma(2,3,5)$, called the \textit{Poincar\'e homology sphere}. 
\end{example}

\subsubsection*{A Brief Tangent on $E_8$}
In this project, we are interested in the properties of the intersection forms of different genus $g$ multisections, especially the minimal $g$ required to get certain forms. It would be fascinating if we could find a genus $2$ multisection whose intersection form is $E_8$, as according to the classification of symmetric bilinear forms, this would tell us a lot about the previous question.

Freedman tells us some essential facts about intersection forms and which manifolds might have such forms. In particular,
\begin{theorem}
	For each symmetric bilinear unimodular form $Q$ over $\Z$, there exists a closed oriented simply-connected topological $4$-manifold with $Q$ as its intersection form. If $Q$ is even, there is exactly one, and if $Q$ is odd, there are exactly two, at least one of which is nonsmoothable\footnote{todo: Cite Morgan Weiler's notes and/or Freedman's paper}.
\end{theorem}
Furthermore, there is a well-known topological manifold whose intersection form is (the Cartan matrix for) $E_8$, called the \textit{$E_8$ manifold}. Given the above theorem and the fact that the $E_8$ manifold has intersection form $E_8$ by construction, this is the unique such manifold. The $E_8$ manifold is constructed as follows: take Freedman's plumbing construction to obtain $M_{E_8}$ such that $\partial M_{E_8}$ is a homology sphere, and glue a fake $4$-ball along this boundary. It turns out that this homology sphere is the Poincar\'e homology sphere\footnote{todo: Cite Morgan Weiler's notes}; this is one of the reasons it is such an important example in this project.

Although we have not yet found Heegaard splittings for general Brieskorn spheres, there is a Heegaard diagram for Poincar\'e that we hope to make use of in the future\footnote{todo: Finish diagram and cite Ozsv\'ath and Szab\'o's PCMI notes}:
\begin{figure}[H]
    \centering
    \incfig[.2]{two-bridge-heegaard-diagram-for-the-poincare-homology-sphere}
    \caption{Two-bridge Heegaard diagram for the Poincar\'e homology sphere}
    \label{fig:two-bridge-heegaard-diagram-for-the-poincare-homology-sphere}
\end{figure}

We are still hoping to pursue these ideas further to create more examples, especially those relating to $E_8$.

\subsection{Generating Examples with Linear Symplectic Geometry}

\subsubsection*{Motivation}
Given that we want to find many examples of multisection diagrams, it would be convenient if there were a way for us to generate compatible cut systems that yield multisections. The ideas for this section come from the observation that, given some Lagrangian subspace $L$ of a symplectic vector space $V$ and some symplectic transformation $T: V \to V$, the image of $L$ under $T$ is again Lagrangian.

It is a natural question then, does the converse hold? That is, between any two Lagrangian subspaces $L,L' \subseteq V$, does there exists a symplectic transformation $T$ such that $T(L) = L'$? If this is the case, then we can consider the seemingly stronger condition that there exists a symplectic map bringing complementary pairs of Lagrangian subspaces to different pairs.

\subsubsection*{Requisite Results}
We first begin with some basic results from linear symplectic geometry over (free) $\Z$-modules. Consider the symplectic module $(V = \Z^{2g}, \omega)$\footnote{todo: Background? How much of this has been covered already?} .
\begin{theorem}
	For any Lagrangian subspace $L_1$ of $V$, there exists a \textit{complementary} or \textit{dual} Lagrangian subspace $L_2$ such that $L_1 \oplus L_2 = V$. Moreover, a choice of basis $(x_1,\dots,x_g)$ of $L_1$ determines a dual basis $(y_1,\dots,y_g)$ for a complement $L_2$ by $\omega(x_i, y_j) = \delta_{ij}$. Taken together, these bases form a \textit{symplectic basis} for $V$. 
\end{theorem}

\begin{proposition}
	If $L_1,L_2$ and $L_1',L_2'$ are pairs of complementary Lagrangian subspaces and $T \in \GL(V)$ such that $L_1 \mapsto L_1'$ and $L_2 \mapsto L_2'$, then $T$ is a symplectic map.
\end{proposition}
\begin{proof}
	Fix a symplectic basis $(x_1,\dots,x_g,y_1,\dots,y_g)$ for $V$ given by the pair $L_1,L_2$. Then it is not hard to see that $T$ induces a symplectic basis $(x_1',\dots,x_g',y_1',\dots,y_g')$ given by $L_1',L_2'$, namely 
\end{proof}


\begin{proposition}
	For any Lagrangian subspace $L$ of $\Z^{2g}$, there exists a unimodular map $T: \Z^{2g} \to \Z^{2g}$ such that $T(L_1) = L$.
\end{proposition}
\begin{proof}
	
\end{proof}

\begin{corollary}
	For any two Lagrangian subspaces $L,L'$ of $\Z^{2g}$, there exists a unimodular map $T$ such that $T(L) = L'$.
\end{corollary}

\begin{proof}
	
\end{proof}

